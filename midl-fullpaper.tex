\documentclass{midl} % Include author names
%\documentclass[anon]{midl} % Anonymized submission

\usepackage{mwe} % to get dummy images
\jmlrvolume{-- Under Review}
\jmlryear{2021}
\jmlrworkshop{Full Paper -- MIDL 2021 submission}
\editors{Under Review for MIDL 2021}

\title[Embedding-based Instance Segmentation of Microscopy Images]{Embedding-based Instance Segmentation \\of Microscopy Images}

\midlauthor{
\Name{Manan Lalit \nametag{$^{1,2}$}} \Email{lalit@mpi-cbg.de}
\AND
\Name{Pavel Tomancak \nametag{$^{2,3}$}} \Email{tomancak@mpi-cbg.de}
 \AND
\Name{Florian Jug \nametag{$^{1,2,4}$}} \Email{jug@mpi-cbg.de} \\
\addr $^{1}$~Center for Systems Biology Dresden (CSBD)\\
\addr $^{2}$~Max Planck Institute of Molecular Cell Biology and Genetics\\
\addr $^{3}$~IT4Innovations, V\v{S}B - Technical University of Ostrava, Ostrava-Poruba, Czech Republic \\
\addr $^{4}$~Fondazione Human Technopole, Milano, Italy}

% Commands...
% --------------------
\newcommand{\EmbedSeg}{\mbox{\textsc{EmbedSeg}}\xspace}
\newcommand{\miniheadline}[1]{\noindent\textbf{#1.}}

\makeatletter
\DeclareRobustCommand\onedot{\futurelet\@let@token\@onedot}
\def\@onedot{\ifx\@let@token.\else.\null\fi\xspace}

\def\eg{\emph{e.g}\onedot} \def\Eg{\emph{E.g}\onedot}
\def\ie{\emph{i.e}\onedot} \def\Ie{\emph{I.e}\onedot}
\def\cf{\emph{c.f}\onedot} \def\Cf{\emph{C.f}\onedot}
\def\etc{\emph{etc}\onedot} \def\vs{\emph{vs}\onedot}
\def\wrt{w.r.t\onedot} 
\def\dof{d.o.f\onedot}
\def\etal{\emph{et~al}\onedot}
\makeatother

\DeclareMathOperator*{\argmax}{arg\,max}
\DeclareMathOperator*{\argmin}{arg\,min}

\begin{document}

\maketitle

\begin{abstract}
Automatic detection and segmentation of objects in microscopy images is important for many biological applications.
In the domain of natural images, and in particular in the context of city street scenes~\cite{Cordts2016Cityscapes}, embedding-based instance segmentation leads to high-quality results.
Inspired by this line of work, we introduce \EmbedSeg, an end-to-end trainable deep learning method based on the work by Neven~\etal~\cite{neven2019}. While their approach embeds each pixel to the centroid of any given instance, in \EmbedSeg, motivated by the complex shapes of biological objects, we propose to use the medoid instead.
Additionally we also make use of a test-time augmentation scheme~\cite{wang2019,stringer2020}, and show that both suggested modifications improve the instance segmentation performance on 2D biological microscopy datasets notably.
 Next, we extend the network architecture of \EmbedSeg to provide a novel, light weight 3D architecture. 
 We demonstrate that our 3D approach achieves competitive results in comparison to state-of-the-art methods on diverse and biologically relevant 3D microscopy datasets.
We also provide two new instance-annotated datasets for use by the community for bench-marking instance segmentation methods.
Lastly, we show that the overall pipeline has a small enough memory footprint to be used on many CUDA enabled laptop hardware.
Our open-source implementation is available at \url{github.com/juglab/EmbedSeg}.
\end{abstract}

\begin{keywords}
List of keywords, comma separated.
\end{keywords}

\section{Introduction}

This is where the content of your paper goes.  Some random
notes\footnote{Random footnote are discouraged}:
\begin{itemize}
\item You should use \LaTeX \cite{Lamport:Book:1989}.
\item JMLR/PMLR uses natbib for references. For simplicity, here, \verb|\cite|  defaults to
  parenthetical citations, i.e. \verb|\citep|. You can of course also
  use \verb|\citet| for textual citations.
\item You should follow the guidelines provided by the conference.
\item Read through the JMLR template documentation for specific \LaTeX
  usage questions.
\item Note that the JMLR template provides many handy functionalities
such as \verb|\figureref| to refer to a figure,
e.g. \figureref{fig:example},  \verb|\tableref| to refer to a table,
e.g. \tableref{tab:example} and \verb|\equationref| to refer to an equation,
e.g. \equationref{eq:example}.
\end{itemize}

\begin{table}[htbp]
 % The first argument is the label.
 % The caption goes in the second argument, and the table contents
 % go in the third argument.
\floatconts
  {tab:example}%
  {\caption{An Example Table}}%
  {\begin{tabular}{ll}
  \bfseries Dataset & \bfseries Result\\
  Data1 & 0.12345\\
  Data2 & 0.67890\\
  Data3 & 0.54321\\
  Data4 & 0.09876
  \end{tabular}}
\end{table}

\begin{figure}[htbp]
 % Caption and label go in the first argument and the figure contents
 % go in the second argument
\floatconts
  {fig:example}
  {\caption{Example Image}}
  {\includegraphics[width=0.5\linewidth]{example-image}}
\end{figure}

\begin{algorithm2e}
\caption{Computing Net Activation}
\label{alg:net}
 % older versions of algorithm2e have \dontprintsemicolon instead
 % of the following:
 %\DontPrintSemicolon
 % older versions of algorithm2e have \linesnumbered instead of the
 % following:
 %\LinesNumbered
\KwIn{$x_1, \ldots, x_n, w_1, \ldots, w_n$}
\KwOut{$y$, the net activation}
$y\leftarrow 0$\;
\For{$i\leftarrow 1$ \KwTo $n$}{
  $y \leftarrow y + w_i*x_i$\;
}
\end{algorithm2e}

% Acknowledgments---Will not appear in anonymized version
\midlacknowledgments{We thank a bunch of people.}


\bibliography{midl-samplebibliography}


\appendix

\section{Proof of Theorem 1}

This is a boring technical proof of
\begin{equation}\label{eq:example}
\cos^2\theta + \sin^2\theta \equiv 1.
\end{equation}

\section{Proof of Theorem 2}

This is a complete version of a proof sketched in the main text.

\end{document}
